\documentclass[11pt, a4paper]{article}
%
\usepackage[utf8]{inputenc}
%
\usepackage[utf8]{inputenc}
\usepackage{tcolorbox}
\usepackage{graphicx}

\newenvironment {myEnv} {\begin{tcolorbox}[width=\textwidth,colback={green}]} {\end{tcolorbox}}
%
\title{Entwicklungsdokument}
\author{Someone}
%
\begin{document}
%   
\maketitle
%
\clearpage
%
\tableofcontents
%
\clearpage
%
\section{Einführung}
% \begin{detailmultibox}{List of units}
%     \begin{detailbox}{unit a}
%         stats\\
%         something\\
%         more\\
%         stuff
%     \end{detailbox}    
%     \begin{detailbox}{unit b}
%         stats\\
%         something\\
%         more\\
%         stuff
%     \end{detailbox}    
%     \begin{detailbox}{unit c}
%         stats\\
%         something\\
%         more\\
%         stuff
%     \end{detailbox}    
% \end{detailmultibox}
% \begin{unitmultibox}{List of units}
%     \begin{unitbox}{unit a}
%         stats\\
%         something\\
%         more\\
%         stuff
%     \end{unitbox}    
%     \begin{unitbox}{unit b}
%         stats\\
%         something\\
%         more\\
%         stuff
%     \end{unitbox}    
%     \begin{unitbox}{unit c}
%         stats\\
%         something\\
%         more\\
%         stuff
%     \end{unitbox}    
% \end{unitmultibox}
%
\section{Hintergrund}
Das Ziel des Projekts ist das Entwickeln einer kleinen Api mit allgemeinen Aufgaben als Programmierübung.
%
\section{Grundidee}
Das Spiel dreht sich um die Besiedlung einer Galaxie.
%
\section{Mechaniken}
In diesem Abschnitt werden grundlegende Mechaniken beschrieben, die das Spiel ausmachen.
%
\subsection{Allgemeines Gameplay}
Das Spiel ist rundenbasiert.
%
\subsection{Planeten}
Planeten verfügen über die folgenden Gesichtspunkte:
\begin{itemize}
	\item Entwicklung - ein Maß für den Ausbau des Planeten
	\item Einrichtungen - spezielle Gebäude die den Planeten nachhaltig verändern
	\item Bevölkerung - die menschlichen Ressourcen des Planeten
	\item Rohstoffe - die natürlichen Ressourcen, die auf einem Planeten vorkommen
	\item Produktion - die verschiedenen Ressourcen, die ein Planet erzeugt, exportiert oder importiert
	\item Verkehr - die Anbindung an interstellare Schiffsrouten
\end{itemize}
%
\subsubsection{Entwicklung}
Entwicklung beschreibt die einerseits die Ausbaustufe des Planeten, andererseits die pro Runde erzeugte
Fähigkeit zum weiteren Ausbau. In jeder Runde steht ein bestimmtes Maß an Entwicklung zur Verfügung.
Dies beschreibt die Fähigkeit den Planeten weiter auszubauen und in bestimmte Richtungen zu entwickeln.
Entwicklung kann auf verschiedene Bereiche verteilt werden. Diese sind:
\begin{itemize}
	\item Wachstum
	\item Wissenschaft
	\item Industrie
	\item Militär
	\item Bauprojekte für Einrichtungen
\end{itemize}
%
Somit kann der Planet in beliebige Richtungen entwickelt werden.
%
\paragraph{Wachstum}
Wachstum führt zu einer Verbesserung der allgemeinen Infrastruktur, des Gesundheitwesens, der kulturellen und sozialen
Einrichtungen des Planeten. Es führt außerdem zum Aufbau neuer Städte und Siedlungen auf dem Planeten und zu einem
Anwachsen der Bevölkerung. Wachstum steht für die Zivile Entwicklung eines Planeten.
%
\paragraph{Wissenschaft}
Wissenschaft führt dazu, dass Ressourcen des Planeten und der Bevölkerung in den wissenschaftlichen Fortschritt
investiert werden. Damit können neue Technologien entwickelt oder entdeckt werden. Wissenschaft kann auch dazu
führen, dass sich andere Aspekte des Planeten verbessern lassen. zum Beispiel durch den Anbau ertragreicherer
Pflanzen, durch das Entwickeln besserer Metalle für das Militär oder die Möglichkeit seltene Rohstoffe zu gewinnen.

\paragraph{Industrie}
Industrielle Entwicklung ermöglicht den Abbau von seltenen und neuen Ressourcen. Es erhöht die Produktion von
Ressourcen und das Potenzial industrielle Produkte zu erzeugen. Eine unkontrolliert wachsende Industrie kann
jedoch das Klima des Planeten negativ beeinflussen.
%
\paragraph{Militär}
Der Ausbau des Militärs ermöglicht die Produktion von besseren Armeeineinheiten und erlaubt die Rekrutierung
von mehr Einheiten. Außerdem wird die Verteidigung des Planeten verbessert. Von der Einlagerung von
Notfallreserven und der Errichtung von Bunkern bis zur Konstruktion von orbitalen Verteidigungsanlagen.
Weiterhin wird die Anzahl der stationierten Streitkräfte und Raumschiffe erhöht.
%
\paragraph{Einrichtungen}
Besondere Einrichtungen benötigen besonders schwerwiegenden Einsatz von Arbeitskraft und Ressourcen.
Für einige besondere Gebäude muss explizit Entwicklung eingesetzt werden. Diese Gebäude verändern das Gesicht
eines Planeten langfristig und unumkehrbar. Sie gewähren große Boni auf spezifische Aspekte des Planeten.
%
\subsubsection{Einrichtungen}
Einrichtungen sind spezielle Gebäude, die einen Planten langfristig und schwerwiegend verändern.
Jeder Planet kann aufgrund ihrer Größe und Bedeutung nur eine einzige Einrichtung unterhalten.
Es existieren die folgenden Einrichtungen:
\begin{itemize}
	\item Orbitaler Ring
	\item Weltuntergangsgerät (Doomsday-Device)
	\item Orbitale Handelsstation
	\item Große Bibliothek
	\item Hydroponische Gärten
	\item Weltenfestung
	\item Heimatwelt für legendäres Regiment
\end{itemize}
%
\paragraph{Orbitaler Ring}
Ein künstlicher Ring aus Metall wird um den Planeten errichtet. Türme, die vom Boden bis in den niedrigen
Orbit reichen versorgen riesige Werften und Kampfstationen mit Material. Der Ring wird in geringem Umfang auch
für Handel genutzt, dient Primär jedoch dem Schiffsbau und der Verteidigung.
%
\paragraph{Weltuntergangsgerät (Doomsday-Device)}
Ein Gerät mit der Kraft ein ganzes Sonnensystem auszuradieren. Durch uralte Technologie wird ein Planet in die
Sonne des Systems gelenkt und verursacht dort massive gravitische Störungen, die nach kurzer Zeit dazu führen,
dass das gesamte Sternensystem von einer Supernova vernichtet wird. Eine Einrichtung die nur im äußersten Notfall
und nur sehr zurückhaltend genutzt wird. Doch dem Feind ein wichtiges System zu überlassen kann mehr Schaden
verursachen als das gesamte Sternensystem mit seiner gesamten Bevölkerung und allen seinen Ressourcen zu
vernichten.
%
\paragraph{Orbitale Handelsstation}
Eine ähnliche Einrichtung wie der Orbitale Ring, nur auf Handel ausgelegt. Die Station macht den Planeten
zu einem Handelszentrum ohne Gleichen in seiner Umgebung. Durch die Technologie der Orbitaltürme können
Tausende Tonnen von Waren und Personal viel effizienter zwischen Orbit und Oberfläche transportiert werden
als mit traditionellen Methoden wie Frachtern.
%
\paragraph{Große Bibliothek}
Eine Sammlung des Wissens der gesamten Rasse. Die große Bibliothek verwandelt den Planeten in einen Ort des
Wissens und der Forschung. Er trägt dazu bei das gesammelte Wissen einer ganzen Rasse zu speichern, zu schützen
und zu vermehren.
%
\paragraph{Hydroponische Gärten}
Der Planet verwandelt sich in einen einzigen, riesigen Garten in dem Früchte aller Art wachsen und gedeihen.
Der Planet wird damit gleichermaßen zu einem Erholungs- und Urlaubsort, sowie zu einer Produktionsstätte für
erlesene Nahrungsmittel. Dadurch lässt sich der Handel eines Planeten verbessern.
%
\paragraph{Weltenfestung}
Der Planet wird in eine Festung verwandelt. Gebirge werden zu riesigen Bunkerkomplexen, Wälder werden abgeholzt
und durch Grabensysteme ersetzt, ganze Landstriche werden vermint. Der Planet wird zur fast uneinnehmbaren
Festung. Um etwas - oder jemanden - zu bewachen oder zu beschützen.
%
\paragraph{Heimatwelt für legendäres Regiment}
Wenn die Truppen eines Planeten eine bestimmte Reputation erreicht haben, gelangt meist auch ihre Heimatwelt
zu gewissem Ruhm. Der Planet, der Krieger mit legendärem Mut, beispielloser Opferbereitschaft oder ungekannter
Grausamkeit hervorbringt wird gleichsam mit Mythen und Legenden versehen. Durch einen Fokus auf die
militärische Tradition einer Welt wird die Qualität der hier ausgebildeten Truppen deutlich erhöht.
%
\subsubsection{Verbesserungen für Sternensystem}
Sternensysteme profitieren von allen Gebäuden und Ausbaustufen der Planeten, die darin enthalten sind.
Außerdem lassen sind in Sternensystemen einzigartige Verbesserungen errichten.
Verbesserungen für Sternensysteme sind
\begin{itemize}
	\item Sternenschmiede
	\item Weltenschiff
\end{itemize}
%
\paragraph{Sternenschmiede}
Unter Ausnutzung der gewaltigen Kräfte des Sterns im Zentrum eines Sternensystems lassen sich extrem seltene
Metalle gewinnen und formen. Eine Sternenschmiede verringert die Lebenszeit eines Sternes und entzieht den
umliegenden Planeten lebenswichtiges Licht ihres Sterns
%
\subsubsection{Rohstoffe}
Rohstoffe stellen die natürlichen Ressourcen eines Planeten dar. Sie werden für die Produktion von Gütern,
den Ausbau des Planeten und den Export benötigt. Jeder Planet besitzt grundlegende Rohstoffe, fortgeschrittene
Rohstoffe und manchmal seltene Rohstoffe. Grundlegende Rohstoffe lassen sich auch mit einfachen Mitteln
nutzen, während für fortgeschrittene Rohstoffe ein bestimmtes Level an industrieller Entwicklung nötig ist.
Für den Abbau seltener Rohstoffe sind zusätzlich noch bestimmte Technologien erforderlich.
\\
Einfache Rohstoffe sind
\begin{itemize}
	\item Anbaufläche
	\item einfache Metalle
	\item einfaches Baumaterial
\end{itemize}
%
Fortgeschrittene Rohstoffe sind
\begin{itemize}
	\item fortgeschrittene Metalle
	\item fortgeschrittene Baustoffe
\end{itemize}
%
Seltene Rohstoffe sind
\begin{itemize}
	\item seltene Metalle
	\item superschwere Isotope
\end{itemize}
%
\paragraph{Anbaufläche}
Die Fläche die auf einem Planeten zum Anbau von Lebensmitteln zur Verfügung steht.
%
\paragraph{einfache Metalle}
%
\paragraph{einfaches Baumaterial}
%
\paragraph{fortgeschrittene Metalle}
%
\paragraph{Fortgeschrittene Baustoffe}
%
\paragraph{seltene Metalle}
%
\paragraph{superschwere Isotope}
%
\subsubsection{Bevölkerung}
Die Bevölkerung eines Planeten stellt neben den natürlichen Rohstoffen die zweitwichtigste Ressource einer Welt
dar. Bevölkerung wird benötigt um Industrie zu betreiben, den Planeten auszubauen und zu bewirtschaften.
Die Bevölkerung eines Planeten lässt sich zu Soldaten ausbilden um ihre Heimat und das Imperium als solches
zu verteidigen und zu vergrößern. Natürlich muss eine Bevölkerung nicht rein menschlich sein. Auf ein und
demselben Planeten können Menschen und andere Lebensformen in Harmonie leben und sich gegenseitig bereichern.
Jedoch trifft dies nicht immer zu. Manche außerirdische Lebensformen sind schlicht zu verschieden voneinander
um friedlich beieinander zu leben. Manche Lebensformen sind durch andauernden Krieg hartherzig geworden und
misstrauisch gegenüber Fremden geworden.\\
Der Zustand der Bevölkerung wird durch die folgenden Aspekte ausgemacht:
\begin{itemize}
	\item Bevölkerungswachstum
	\item Arbeit
	\item Rekrutierung
	\item Unruhe
	\item Rebellion
\end{itemize}
%
\paragraph{Bevölkerungswachstum}
Durch die Investition in die Entwicklung "Wachstum" werden Bereiche des öffentlichen Lebens gefördert, die
ein Wachstum der Bevölkerung fördern. Bessere Medizin, mehr Nahrung, Bildung, Arbeitsplätze und allgemein
gesteigerter Wohlstand sind Faktoren die zu einer Vermehrung der Bevölkerung führen.
Schlechte Lebensbedingungen, Nahrungsmittelknappheit, Seuchen, Umweltverschmutzung oder Krieg können die
Bevölkerung eines Planeten dezimieren.
%
\paragraph{Arbeit}
Die Bevölkerung eines Planeten ist entscheidend für die Arbeitskraft des Planeten. Je größer die Bevölkerung,
desto mehr Kapazitäten stehen zur Verfügung um Maschinen zu bedienen, Rohstoffe zu verarbeiten, Pflanzen
anzubauen und andere Arbeiten zu verrichten. Ohne eine ausreichend große Bevölkerung kommt die Produktion
eines Planeten zum Erliegen.
%
\paragraph{Rekrutierung}
Die Bevölkerung eines Planeten stellt nicht nur Arbeitskräfte, sondern auch Soldaten und militärisches
Personal. Um eine schlagkräftige Armee aufzustellen wird nicht nur Material benötigt, sondern auch eine große
Bevölkerung.
%
\paragraph{Unruhe}
Wenn sich die Zustände auf einem Planeten verschlechtern, sei es durch Luftverschmutzung, Überbevölkerung,
Krankheiten, Krieg ode Rebellion auf anderen Welten, kann die Stimmung der Bevölkerung bedrückt werden.
Eine unzufriedene Bevölkerung neigt zur Unruhe und zu Protesten. Ist die Bevölkerung unruhig ist sie anfälliger
für das wirken von Kultisten, von feindlichen Infiltratoren und Rebellen. Unruhe verringert die Produktion
eines Planeten leicht.\\
Manche Unruhen legen sich von selbst, wenn die Bevölkerung sich an ihre Lebensbedingungen gewöhnt, andere
entladen sich in gewaltsamen Protesten und können schließlich zur offenen Rebellion führen.
%
\paragraph{Rebellion}
Ist die Bevölkerung eines Planeten über längere Zeit unruhig, so kann es zu einer Rebellion kommen. Rebellion
beginnt mit gewaltsamen Protesten und Aufständen. Wenn diese nicht schnell und hart niedergeschlagen werden,
versinkt der Planet schnell im Bürgerkrieg. Je nachdem ob Rebellen oder Loyalisten (sofern noch loyale Kräfte
vorhanden sind) diesen Krieg gewinnen, fällt der Planet wieder dem Imperium zu oder sagt sich von diesem los.\\
Während einer aktiven Rebellion ist die Produktion eines Planeten stark reduziert. Während einer Rebellion kann
der Planet nicht kontrolliert werden und die Infrastruktur des Planeten wird beschädigt.\\
%
\subsection{Militär}
Ein weiterer Hauptaspekt des Spiels ist das Militär. Das Militär, das einem Spieler zur Verfügung steht
gliedert sich in Garnisonen und Armeen. Weiterhin zählen auch Verteidigungsanlagen zum Militär.
Garnisonen und Armeen bestehen jeweils aus einer Raumflotte und/oder Bodentruppen. Prinzipiell kann jeder
Planet Truppen ausheben. Jedoch wird die Verfügbarkeit bestimmter fortgeschrittener Truppen durch benötigte
Ressourcen und Technologien eingeschränkt.
%
\subsubsection{Rekrutierung}
Rekrutierung stellt den ersten Schritt zum Aufbau einer Streitmacht dar. Vom einfachen Fußsolaten, über
Panzerbatallione, bis hin zu mächtigen Raumschiffen benötigt jede militärische Einheit Personal und Rohstoffe.
Meist werden zusätzlich noch bestimmte Technologien benötigt um eine Einheiten auszubilden. Prizipiell können
alle Truppentypen auf jedem Planeten ausgehoben werden. Dafür müssen folgende Vorraussetzungen erfüllt sein:
\begin{itemize}
	\item der Planet verfügt über alle erforderlichen Technologien
	\item der Planet verfügt über ausreichend wissenschaftliche Entwicklung um die Technologie(n) anzuwenden
	\item es stehen ausreichend Ressourcen zur Verfügung
	\item die Bevölkerung ist ausreichend groß um Rekruten anzuwerben
\end{itemize}
%
Sowohl Qualität als auch Quantität der rekrutierten Einheiten hängen stark von der militärischen Entwicklung
eines Planeten ab. Da Rekrutierung immer Teile der normalen Bevölkerung in Soldaten und militärisches Personal
umwandelt, kann Rekrutierung auch gezielt genutzt werden um das Bevölkerungswachstum eines Planeten zu
kontrollieren. Hat ein Planet beispielsweise einen Bevölkerungsüberschuss können dadruch Unruhen entstehen.
Indem Millionen einfacher Bürger zwangsrekrutiert werden, lässt sich die Bevölkerung entscheidend reduzieren
und gleichzeitig die Garnison aufbessern, falls es zu einer Rebellion kommt.
%
\subsubsection{Garnison}
Jeder Planet verfügt über ein stehendes Heer, das meist aus einfacher Infanterie besteht und von gepanzerten
Einheiten unterstüzt wird. Diese Garnison ist fähig Rebellionen nieder zu schlagen und den Planeten gegen
Piraten und Plünderer zu halten, kann aber einer groß angelegten Invasion eines Gegners meist wenig
entgegensetzen. Durch eine militärische Entwicklung wird die Größe und Qualität der Garnison eines Planeten
verbessert.
%
\subsubsection{Verteidigungsanlagen}
Neben einer Garnison verfügen die meisten Planeten über stationäre Verteidigungsanlagen. Eine
Verteidigungsanlage kann beinahe jede vorstellbare Form annehmen. Von primitiven Minenfelder auf der
Oberfläche über Bunker, Luftabwehrgeschütze, Anti-Orbital-Raketen bis hin zu Kampfstationen im Orbit.
Durch eine militärische Entwicklung erhält der Planet mehr und bessere Verteidigungsanlagen.
Orbitale Verteidigungsanlagen stellen die erste Linie der Verteidigung eines Planeten dar. Jeder Gegner muss
sie entweder komplett zerstören, entern oder mit schnellen Schiffen durchbrechen.
%
\subsubsection{Bodentruppen}
Bodentruppen sind die einfachen (oder fortschittlichen) bewaffneten Männer und Frauen des Imperiums.
Die Unterschiede in der Qualität von Bodentruppen ist gewaltig. Von einfachen Soldaten mit einem massenweise
produzierten Gewehr, die kaum gepanzert zu Tausenden in die Schlacht ziehen, bis hin zu schwer bewaffneten,
Elitekriegern in mechanisierten Rüstungen ist alles möglich. Einige Soldaten werden in massiven Wellen gegen
den Feind geworfen und überwältigen den Gegner durch schiere Zahlen. Ander Einheiten von höherer Qualität
können einen Krieg schon in kleinen Zahlen gewinnen. Entweder weil sie aus dem Untergrund zuschlagen, den
Feind infiltrieren, Versorgungswege abschneiden, Fabriken sabotieren, Unruhe in der Bevölkerung stiften oder
gegnerische Anführer eliminieren. Oder sie kämpferische Fähigkeiten ohne Gleichen haben und leicht hundert
normale Soldaten bezwingen können.\\
Bodentruppen stellen in all ihrer Vielfalt das Grundgerüst aller Streitkräfte dar. Jeder Planet kann
mindestens eine Art dieser Basistruppen ohne besondere Anforderungen ausbilden.
%
\subsubsection{Fahrzeuge}
Fahrzeuge fassen alle mechanisierten Militäreinheiten zusammen. Dabei kann es sich um schnelle Motorräder zur
Aufklärung, schwere Kampfpanzer, Bomber, oder auch fliegende Truppentransporter handeln. Fahrzeuge benötigen
wesentlich mehr Ressourcen als vergleichbare Bodentruppen. Fahrzeuge können entweder direkt als Kampffahrzeuge
genutzt werden oder als Transporter, um Bodentruppen die Schlacht zu transportieren oder Verwundete zu
evakuieren. Oft ist es sinnvoll wertvolle Eliteeinheiten in Transportern unterzubringen um sicherzustellen,
dass sie die Frontlinien erreichen.\\
In vielen Schlachten sind Flugzeuge von außerordentlicher Bedeutung, da die Lufthoheit der erste Schritt zum
Sieg ist. Fahrzeuge können, wie Bodentruppen, Teil einer Armee oder einer Garnison sein. Sie können auch in
serselben Art auf Raumschiffen transportiert werden.
%
\subsubsection{Raumschiffe}
Raumschiffe stellen die wichtigsten Einheiten im Spiel dar. Raumschiffe werden primär benötigt um Bodentruppen
und Fahrzeuge auf gegnerische Planeten zu transportieren um eine Invasion zu starten. Unabhängig davon sind
viele Raumschiffe auch auf den Kampf gegen feindliche Raumschiffe oder orbitale Verteidigungsanlagen
ausgelegt. Andere Raumschiffe sind eher dazu geeignet feindliche Verteidigungslinien zu durchbrechen und
Elitetruppen direkt auf einem Planeten abzusetzen.\\
Raumschiffe benötigen fortschrittliche Technologie und erheblichen Einsatz von Arbeitskräften und Material.
Doch die Kosten lassen sich rechtfertigen, da es keine andere Möglichkeit gibt fremde Planeten zu kolonisieren.
Raumschiffe lassen sich in folgende Kategorien einteilen:
\begin{itemize}
	\item Frachter
	\item Makrofrachter
	\item Unterstützungsschiffe
	\item Entdecker
	\item Jäger
    \item Bomber
    \item Truppentransporter
	\item Brander
	\item Kreuzer
	\item Fregatten
	\item Zerstörer
	\item Träger
	\item Schlachtschiffe
\end{itemize}
%
\paragraph{Frachter}
Frachter stellen eine grundlegende aber sehr bedeutsame Schiffsform dar. Frachter werden primär dazu genutzt
um Personal, Güter und Militäreinheiten zu transportieren. Frachter werden nicht nur militärisch genutzt,
sonder überwiegend für Handel. In einer militärischen Raumflotte sind Frachter jedoch ebenso unerlässlich wie
die eigentlichen Kriegsschiffe. Sie transportieren Bodentruppen und Panzer - viel wichtiger jedoch -
Nahrung, Treibstoff und Munition, Waffen und medizinische Vorräte. Sie spielen auch eine wichtige Rolle, wenn
es darum geht einen Planeten zu plündern.\\
Kämpferisch sind Frachter meist schlecht aufgestellt. Eine Panzerung und einige leichte Waffen zum Schutz
gegen Piraten sind vorhanden, jedoch hat ein einfacher Frachter keine Chance gegen ein echtes Kriegsschiff.
%
\paragraph{Makrofrachter}
Der große Bruder des einfachen Frachters. Diese Giganten machen selbst Schlachtschiffen Konkurrenz, wenn es um
schiere Größe und Motorisierung geht. Ein einzelner Makrofrachter ist in der Lage eine Jahresproduktion einer
ganzen Welt oder eine Armee mit allen nötigen Vorräten aufzunehmen. Entsprechend besser sind diese Schiffe
gepanzert und bewaffnet. Da die schiere Menge ihrer Fracht sie auch zu lohnenden Zielen macht, verfügen sie
meist über besonders verstärkte Rümpfe und verbesserte Schilde. Im Kampf gegen leichte Kriegsschiffe haben
Makrofrachter durchaus eine Chance, jedoch ist eine Flucht immer die bevorzuge Taktik um die wertvolle Fracht
zu schützen.
%
\paragraph{Unterstützungsschiffe}
Jede größere Flotte führt ein Kontingent von Unterstützungsschiffen mit sich. Diese können verschiedene Rollen
übernehmen. Einige Schiffe können Reparaturen an anderen Schiffen durchführen, andere Sammeln und Verwerten
Trümmerteile nach einer Schlacht. Wieder andere werden als Späher eingesetzt und liefern präzise Zieldaten
für die Waffen der größeren Schiffe. Allgemein werden diese Schiffe nie im direkten Kampf eingesetzt, da sie
schwach gepanzert und nur selten bewaffnet sind.
%
\paragraph{Entdecker}
Entdeckerschiffe haben eine einfache Aufgabe: neue Planeten und interessante Orte im Universum erkunden.
Entdeckerschiffe sind mittelgroße Schiffe, die primär mit Scannern, großen Vorratskammern und starken 
Triebwerken ausgerüstet sind. Sie durchstreifen die Weiten des Alls auf der Suche nach koloniesirbaren
Welten, neuen Handelspartnern, Rohstoffvorkommen und den Wundern des Universums. Da nie gewiss ist, wie andere
Rassen auf Neuankömmlinge reagieren werden sind Entdecker für ihre Größe gut gepanzert und bewaffnet.
%
\paragraph{Jäger}
Jäger sind kleine raumfähige Schiffe, die von Planeten, orbitalen Basen oder größeren Kampfschiffen aus 
gestartet werden. Ihre Hauptaufgabe ist es feindliche Jäger und Bomber auszuschalten. Sie sind jedoch auch 
dazu in der Lage wesentlich größere Schiffe zu beschädigen. Jäger agieren meist in Geschwadern von mehreren 
Schiffen. Sie lassen sich von einem einzelnen Piloten oder von einer kleinen Crew steuern und ihre größte 
Stärke sind ihre Geschwindigkeit und ihre geringe Größe. Während Jäger ein hohes offensives Potenzial haben
sind sie schlecht gepanzert und verlassen sich darauf gar nicht erst getroffen zu werden.
%
\paragraph{Bomber}
Bomber sind nur geringfügig größer als Jäger und lassen sich, je nach Größe, von einer Crew von 3-10 Mann 
gut steuern. Bomber sind ähnlich schnell, wendig und schlecht gepanzert wie Jäger, sind im Gegensatz zu diesen
jedoch sehr stark bewaffnet. Ihre Waffen sind dabei nicht auf den Kampf gegen feindliche Jäger ausgelegt, 
sondern sollen die größeren Kampfschiffe angreifen. Sie tragen primär Torpedos, Haftbomben oder andere 
Hochexplosive Waffensysteme. Ein einziger Bomber, der sich einem gegnerischen Schlachtschiff unbemerkt nähert
kann so den Antrieb oder einen Schildgenerator zerstören und den Gegner lähmen bis die eigenen Schiffe sich
in Stellung gebracht haben. Allzu oft endet die Mission eines Bombers jedoch im Feuer gegnerischer 
FLAK-Geschütze oder Jäger.
%
\paragraph{Truppentransporter}
Truppentransporter sind ähnlich groß wie Bomber, meist etwas größer. Sie tragen kaum Waffen, sind dafür aber
besser gepanzert und schneller als Jäger. Ihre Aufgabe besteht entweder darin Entertruppen auf ein feindliches
Schiff zu transportieren oder Bodentruppen vom Orbit auf die Oberfläche eines Planeten.
%
\paragraph{Brander}
Brander stellen ein antikes und verzweifeltes, aber höchst effektives Mittel der Kriegsführung dar. Dabei 
werden alte oder schwer beschädigte Schiffe mit Sprengstoffen beladen und direkt in gegnerische Schiffe 
gelenkt. 
%
\paragraph{Kreuzer}
Kreuzer stellen die kleinste Klasse der Kriegsschiffe dar. Sie sind schnell, gut bewaffnet und ausreichend 
gepanzert um es mit anderen Kreuzern aufnehmen zu können. Ihre Hauptaufgabe besteht darin feindliche Linien
zu durchbrechen oder zu umgehen und die Unterstützungsschiffe und Träger des Gegners anzugreifen. Kreuzer
können auch in kleinen Verbänden eingesetzt werden um gegnerische Handelsrouten zu störe oder eigene zu
schützen. Sie werden auch oft als Eskorten für wichtige Personen eingesetzt.\\
Kreuzer sind eine der flexibelsten Schiffsklassen. Sie können je nach Einsatzzweck mit Torpedos, Lanzenwaffen,
als Blockadebrecher oder zur Abwehr von Jägern und Bombern ausgestattet werden.
%
\paragraph{Fregatten}
Fregatten stellen die kleinste Schiffsklasse dar, die alleine operieren kann. Fregatten werden jedoch meist 
von einem kleinen Geschwader Kreuzer begleitet. Sie sind besser gepanzert und bewaffnet als Kreuzer, jedoch 
etwas langsamer. Sie nehmen die Rolle eines kleinen Linienschiffes ein und bekämpfen primär Gegner von ihrer
Größe oder wehren Kreuzer ab. Dabei kann es eine einzelne Fregatte durchaus mit mehreren Kreuzern gleichzeitig
aufnehmen und siegreich sein. Fregatten sind eine sehr populäre Schiffswahl, da sie gute Kämpfer und 
gleichzeitig sehr günstig herzustellen sind.
%
\paragraph{Zerstörer}
Zerstörer sind kleine, wendige Schiffe die Panzerung zugunsten Geschwindigkeit und Schlagkraft aufgeben.
Sie sind meist mit Torpedos und Lanzenbatterien bewaffnet und können so selbst Schlachtschiffen empfindlichen
Schaden zufügen. Sie eignen sich hervorragend um Kreuzer oder Fregatten auf lange Reichweite zu bekämpfen,
haben jedoch auf mittlere und kurze Reichweite der Offensivkraft dieser Schiffe nicht mehr viel 
entgegenzusetzen. Daher begleiten Zerstörer meist die größeren Kriegsschiffe und schlagen aus großer 
Entfernung zu.
%
\paragraph{Träger}
Träger sind ähnlich groß wie Schlachtschiffe. Sie sind extrem stark gepanzert und verfügen meist über 
zusätzliche Energieschilde. Ihre Bewaffnung besteht hauptsächlich aus Kurzstreckenwaffen zur Abwehr kleinerer
Schiffe. Diese Schiffe sollten nie in einer direkten Konfrontation mit Schlachtschiffen geraten, ihre Aufgabe 
besteht primär darin Jäger und Bomber in die Schlacht zu transportieren. Ein Großteil des Schiffsraumes wird 
daher von Hangarbuchten, Reparaturwerften, Treibstofftanks und Munitionslagern eingenommen.
%
\paragraph{Schlachtschiffe}
Schlachtschiffe sind die Könige des Raumkampfes. Schwer bewaffnet, schwer gepanzert und zusätzlich durch 
Energieschilde geschützt stellen diese Kolosse meist den Ankerpunkt einer Flotte dar. Häufig nutzen Offiziere
genau diese riesigen Schiffe als Gefechtsleitstand. Der einzige Nachteil dieser Schiffe ist ihre geringe
Geschwindigkeit. Dafür sind sie jedoch außerordentlich robust und schlagkräftig. Ein kleineres Schiff besteht 
selten länger als ein paar Augeblicke nachdem es von einem einem Schlachtschiff erfasst wurde.\\
Es ist ein äußerst seltener Anblick mehrere Schlachtschiffe in einer Flotte zu sehen und ein Zeichen dafür wie
verzweifelt die Situation oder wie mächtig ein Gegner ist, wenn mehrere solche Schiffe gleichzeitig eingesetzt
werden. Schlachtschiffe sind meist alt, allein ihr Bau dauert meist mehrere Jahrzehnte. Sie sind meist 1000
Jahre oder länger im Einsatz, sofern sie nicht im Kampf kritischen Schaden erleiden. Aufgrund ihrer 
gigantischen Ausmaße und der meterdicken Panzerung ist es jedoch fast unmöglich diese Schiffe ganz zu 
zerstören. Oft lassen sich die Überreste solcher Schiffe jedoch entweder reparieren oder recyclen.
%
\subsection{Kampf}
Das Kampfsystem ist ein zentraler Aspekt des Spiels. Es gibt zwei Arten von Kampf. Einerseits den Kampf um 
einen Planeten, eine planetare Invasion und andererseits den Raumkampf, bei dem zwei Flotten im leeren Raum
aufeinander treffen. Der Raumkampf gliedert sich in die folgenden Phasen:
\begin{itemize}
    \item Kampf auf lange Reichweite
    \item Kampf auf mittlere Reichweite
    \item Kampf auf kurze Reichweite
\end{itemize}
%
Der Kampf um einen Planeten hat folgende Phasen, die in verschiedenen Reihenfolgen oder auch gleichzeitig
stattfinden können.
\begin{itemize}
    \item Kampf gegen lokale Flotten und die orbitale Verteidigung
    \item Zerstörung der Antiorbitalgeschütze auf der Oberfläche
    \item Bombardierung des Planeten aus dem Orbit
    \item Landung von Bodentruppen und Fahrzeugen
    \item Krieg am Boden und in der Luft
    \item Infiltration des Planeten
\end{itemize}
%
\end{document}
